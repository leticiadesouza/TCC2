\begin{resumo}[Abstract]
 \begin{otherlanguage*}{english}
   The University of Brasília (UnB), through the Information Technology Research and Application Center (ITRAC), in partnership with the Ministry of Economy (ME), has contributed with the Brazilian Federal Government for the transformation of services public services in digital services. The ITRAC defined a systematized validation process with the purpose of guaranteeing the quality of these services, based on the use of the Exploratory Tests, with the Tourist Metaphor, which depends on the personality and knowledge of those involved in the test activity. Thus, it is assumed that the tester profile has some impact on the application of the \ textit {tours} used in the tourist metaphor. In this work, we propose to obtain profiles of testers, using personal characteristics and domain experience for automatic assignment of test activities, from the development of a recommendation system, using the team of ITRAC testers. The main technical procedure chosen for this evaluation was Research-Action, commonly used in solving collective problems, involving researchers and participants in a cooperative way. The first stage of this methodology, the Diagnosis, was carried out to understand the object of study, while the last three steps, Planning, Action, and Evaluation will be taken into account for the construction of instruments in cycles of interaction with the team, in order to facilitate the understanding of the operation of the module of the proposed tool.

   \vspace{\onelineskip}

   \noindent
   \textbf{Key-words}: System of Recommendation. Exploratory Testing. Profile of Testers.
 \end{otherlanguage*}
\end{resumo}
