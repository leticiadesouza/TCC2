\chapter{Resultados}
\label{ch:resultados}

\section{Ameaça à Validade}

No trabalho de \cite{wohlin2012experimentation}, é utilizado um esquema de classificação proposto por \cite{yin2017case} para estudos de caso e, similar a isso, um esquemático foi elaborado para a engenharia de software. No esquema citado são definidas quatro aspectos de validação, como se segue:

\begin{itemize}
    \item \textit{Construct Validity}: 
    \item \textit{Internal Validity}: 
    \item \textit{External Validity}: 
    \item \textit{Reliability}: 
\end{itemize}

A estratégia inicial para a coleta das informações sobre o perfil dos testadores, foi traçada para ser realizada a partir da execução da proposta deste trabalho com a equipe de testadores do Ministério da Economia. Entretanto, devido a uma troca de governo e reorganização de diretorias, o serviço de testes dos serviços digitizados foi retirado das prioridades do ME. 

Por conta disso, a execução do experimento foi redelineada para ser executada em duas disciplinas do curso de Engenharia de Software, a primeira, Testes de Software, possui alunos a partir do 5º semestre, a segunda, Engenharia de Software Experimental, possui alunos a partir do 8º semestre do curso.

\section{Experimentos com a Ferramenta de Transformação Digital}

Para realizar o experimento proposto neste trabalho, foi elaborado um serviço fictício na ferramenta da \textit{Lecom} com o tema de \textit{Agendamento para Visitas ao Senado [teste]}, vide \ref{anexo:servicoFicticio}. O objetivo da criação do serviço fictício foi projetar defeitos ao longo de todo o serviço, de modo que este pudesse ser repassado aos testadores para que eles testassem e encontrassem o máximo de defeitos possível, utilizando as \textit{tours} pertencidas à Metáfora do Turista.

Para que os alunos realizassem o teste, foi criado um Guia para Aplicação de Testes Exploratórios, como pode ser verificado no Anexo \ref{anexo:guiaTestesExploratorios}. Os alunos das disciplinas de Testes de Software e de Engenharia de Software Experimental, tiveram acesso a um material que continha um resumo das \textit{tours}, que se encontra no Anexo \ref{anexo:resumoTours}, que foi passado em sala de aula previamente, juntamente com um aula sobre testes exploratórios, para que os estudantes obtivessem uma bagagem prévia sobre a abordagem. 

Para 