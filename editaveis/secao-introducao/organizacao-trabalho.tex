\section{Organização do Trabalho}

Este trabalho de conclusão de curso está organizado nos seguintes capítulos:

\begin{itemize}
    \item \textbf{Capítulo \ref{ch:introducao} - Introdução:} apresentou Contextualização, justificativa, problema de pesquisa, justificativa e objetivos;
    \item \textbf{Capítulo \ref{ch:referencial} - Referencial teórico:} descreve os conceitos que fundamentam o trabalho reunindo conhecimento necessário para que se compreenda a pesquisa realizada. O capítulo é subdividido nas seções \textit{Governo Digital}, \textit{Testes Exploratórios} e \textit{Sistema de Recomendação};
    % \item \textbf{Capítulo \ref{ch:suporte} - Suporte tecnológico:} apresenta as ferramentas que suportarão as atividades de desenvolvimento de software, gerenciamento, documentação, dentre outras.
     \item \textbf{Capítulo \ref{ch:metodologia} - Materiais e Métodos:} apresenta o plano metodológico adotado e caracteriza o objeto de estudo;
     \item \textbf{Capítulo \ref{ch:proposta} - Proposta:} Apresenta a proposta deste trabalho, desde a abordagem até a proposta de um módulo para a ferramenta de suporte ao processo de transformação digital do governo brasileiro.
     
    % \item \textbf{Capítulo \ref{ch:resultados} - Resultados parciais:} apresenta os resultados alcançados durante o TCC1.
    
    % \item \textbf{Capítulo \ref{ch:consideracoes} - Considerações finais:} relata o status do trabalho alcançado até a execução do TCC1 e os resultados esperados para o TCC2.
\end{itemize}